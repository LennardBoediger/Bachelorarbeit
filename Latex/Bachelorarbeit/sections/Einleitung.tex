Es gibt viele wissenschaftliche und industrielle Anwendungen, die zuverlässige Daten über die Spektralezusammensetzung des menschlich sichtbaren Tageslichts erfordern.\\
Das menschlich sichtbare Licht besteht aus den Wellenlängen der elektromagnetischen Strahlung ($380 - 780 nm$), die ursprünglich von unserer Sonne emittiert werden und direkt oder indirekt die Erde erreichen.
Um das Licht über den gesamten sichtbaren Bereich des Himmels zu erfassen, sind mehrere Sensoren zur Messung in verschiedenen Richtungen oder mechanische Komponenten zur Abtastung des Himmels in mehreren Punkten erforderlich.\\
Das Fachgebiet Lichttechnik benutzt bereits einen Messaufbau. Dieser verwendet einen Spektrometer der durch einen beweglichen Spiegel und eine rotierbare Basis in der Lage ist den Himmel abzutasten. 
Der Messaufbau kann somit mit einer sehr genauen spektralen Auflösung von 1 nm in 145 Himmelsbereichen messen.\\
Im Dauereinsatz haben sich mechanische Komponenten unter dem Einfluss äußerer Bedingungen als gravierender Schwachpunkt erwiesen.\\
 Die Messeinrichug ist aufgund des hochwertigen Sensors in der Lage sehr genaue Messungen aufzuzeichnen, die Anschaffung ist jedoch mit erheblichen Kosten verbunden.\\
Aus diesem Grund wird am Fachgebiet für Lichttechnik der Technischen Universität Berlin an einer transportablen, spectral- und richtungsauflösenden Messeinrichtung geforscht die ohne mechanische Komponeten auskommt und durch die Verwendung von weniger genauen Sensoren eine kostengünstige und robuste Alternative darstellt.

\noindent In dieser Arbeit wird mithilfe der von AMS entwickelten auf Fotodioden basierenden Lichtsensoren  AS7261 und AS7265X eine Spektral-Sensor Plattform entwickelt, die Modular für unterschiedliche Forschungszwecke, aber vor allem zur Tageslichtmessung, genutzt werden kann.


\subsection{Aufbau des Messystems}
Ein Einplatinencomputer wird mit bis zu Zwölf AS7261 und Zwölf AS7265X Sensoren verbunden.
Die Messungen werden über eine I2C Bus Verbindung gesteuert und ausgelesen.
Auf dem Einplatinencomputer  wird eine Datenbank mit den Messdaten gefüllt und über ein Webinterface welches auch auf dem Einplatinencomputer gehostet wird verfügbar gemacht.
